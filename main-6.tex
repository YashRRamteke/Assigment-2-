\documentclass[journal,12pt,twocolumn]{IEEEtran}

\usepackage[utf8]{inputenc}
\usepackage{graphicx}
\usepackage{amsmath}
\usepackage{mathrsfs}
\usepackage{txfonts}
\usepackage{stfloats}
\usepackage{bm}
\usepackage{cite}
\usepackage{cases}
\usepackage{subfig}
\usepackage{amsfonts}
\usepackage{amssymb}
\usepackage{enumitem}
\usepackage{mathtools}
\usepackage{tikz}
\usepackage{circuitikz}
\usepackage{verbatim}
\usepackage[breaklinks=false,hidelinks]{hyperref}
\usepackage{listings}
\usepackage{calc}
\usepackage{float}
\usepackage{longtable}
\usepackage{multirow}
\usepackage{multicol}
\usepackage{color}
\usepackage{array}
\usepackage{hhline}
\usepackage{ifthen}
\usepackage{chngcntr}


\title{\textbf{\underline{Assignment 2}}}
\author{\textbf{\textit{Yash R Ramteke}}\\
\textbf{\textit{bt21btech11006}}}

\begin{document}

\maketitle

\textbf{Question:}
Determine the binomial distribution where mean is $9$ and standard deviation is $\frac{3}{2}$ Also, find the probability of obtaining at most one success.

\bigskip
\textbf{Solution:}
For binomial distribution :\\
Given, Mean = $9$ and Standard Deviation(S.D) = $\frac{3}{2}$
\begin{align}
    \text{Mean = np} = 9
    \\
    \text{Variance} = (\text{S.D.})^{2} = \text{npq} = \frac{9}{4}
\end{align}
By substituting equation($1$) in equation($2$):
\begin{align}
      \text{q} = \frac{1}{4}
\end{align}
Since, p = $1 -$q
\begin{align}
      \text{p} = 1 - \frac{1}{4} = \frac{3}{4} 
\end{align}\\
Using equation ($4$) in equation ($1$):
\begin{align}
      \text{n} = \frac{9}{p} = \frac{4\times9}{3} = 12
\end{align}\\
Thus distribution is:\\
 Let $X\in\{0,1\}$ \\
$X\sim \text{Bin(n,p)}\sim \text{Bin(m,p)}$\\ 
\text{Now let} $0\leq k \leq (n+m),\text{ then}$\\
\begin{align}
 P(X) = \sum_{i=0}^{k}P(i,k-i)
\end{align}
\begin{align}
\sum_{i=0}^{k}\left(\begin{array}{ll}
    n\\
    i
    \end{array}\right){p}^i(1-p)^{n-i} \left(\begin{array}{ll}
    m\\
    k-i
    \end{array}\right)p^{k-i}(1-p)^{m-k+i}
\end{align}
\begin{align}
    &= p^k(1-p)^{n+m-k}\sum_{i=0}^{k}\left(\begin{array}{cc}
    n\\
    i
    \end{array}\right)\left(\begin{array}{ll}
    m\\
    k-i
    \end{array}\right)\\
&= \left(\begin{array}{cc}
    n+m\\
    k
    \end{array}\right) \text{p}^k \left(1-\text{p}\right)^{n+m-k}
\end{align}
\begin{align}
\text{P(k=r)} &= ^{12}C_r (\text{p})^r (\text{q})^{12-r}\\
\text{P(k=r)} &= ^{12}C_r \left(\frac{3}{4}\right)^r \left(\frac{1}{4}\right)^{12-r}
\end{align}

\begin{center}
     \text{r = $0,1,2,3...$}
\end{center}
P(at most one success) = P(k=0) + P(k=1)
\begin{align}
&= ^{12}C_0 \left({\frac{3}{4}}\right)^0 \left(\frac{1}{4}\right)^{12} + ^{12}C_1 \left(\frac{3}{4}\right)^1 \left(\frac{1}{4}\right)^{11} 
\\
&= \left(\frac{1}{4}\right)^{12} + 36\left(\frac{1}{4}\right)^{12} =  \frac{37}{4^{12}}
\end{align}

\end{document}
